\documentclass{bmstu}
\usepackage{multirow}
\usepackage{threeparttable}
\usepackage[clean]{svg}
\usepackage{adjustbox}

% пакеты
\usepackage{csquotes}
\usepackage[utf8]{inputenc}
\usepackage[T1]{fontenc}

% Библиография
\NewBibliographyString{accessmode}
\DeclareFieldFormat{url}{\bibstring{accessmode}\addcolon\space\url{#1}}
\DeclareFieldFormat{title}{#1}
\DefineBibliographyStrings{russian}{
accessmode={Режим доступа:},
urlseen={дата обращения:}
}
% скобки после пунктов
\setenumerate[1]{label={ \arabic*)}}
\usepackage{enumitem}

% математика в листингах
\lstset{
mathescape=true
}

% \FloatBarrier
\usepackage{placeins}
\usepackage{svg}

% $ в листингах
\newcommand{\dlr}{\mbox{\textdollar}}

% Тире в itemize
\renewcommand\labelitemi{---}

% \foreach
\usepackage{pgffor}

% Длинные таблицы
\usepackage{longtable}

% Картинки
\newcommand{\img}[3] {
\begin{figure}[t!]
    \center{\includegraphics[height=#1]{inc/img/#2}}
    \caption{#3}
    \label{img:#2}
\end{figure}
}
\newcommand{\imgw}[3] {
\begin{figure}[h!]
    \center{\includegraphics[width=#1]{inc/img/#2}}
    \caption{#3}
    \label{img:#2}
\end{figure}
}
\newcommand{\boximg}[3] {
\begin{figure}[h]
    \center{\fbox{\includegraphics[height=#1]{inc/img/#2}}}
    \caption{#3}
    \label{img:#2}
\end{figure}
}

% АА в сокращенном виде

\newcommand{\aasection}[2] {
    \vspace{20mm}
    {\let\clearpage\relax \chapter{#1}\label{#2}}
}

\newcommand{\aaunnumberedsection}[2] {
    \vspace{20mm}
    \addcontentsline{toc}{chapter}{#1}
    {\let\clearpage\relax \chapter*{#1}\label{#2}}
}




\bibliography{biblio}

\begin{document}
\renewcommand{\thelstlisting}{\arabic{lstlisting}}
\lstset{%
	language=python,
	basicstyle=\small\sffamily,
	numbers=left,
	stepnumber=1,
	numbersep=5pt,
	frame=single,
	tabsize=4,
	captionpos=t,
	breaklines=true,
	breakatwhitespace=true,
	escapeinside={\#*}{*)},
	backgroundcolor=\color{white},
}

\begin{titlepage}
	\newgeometry{pdftex, left=2cm, right=2cm, top=2.5cm, bottom=2.5cm}
	\fontsize{12pt}{12pt}\selectfont
	\noindent \begin{minipage}{0.15\textwidth}
		\includegraphics[width=\linewidth]{img/b_logo.jpg}
	\end{minipage}
	\noindent\begin{minipage}{0.9\textwidth}\centering
		\textbf{Министерство науки и высшего образования Российской Федерации}\\
		\textbf{Федеральное государственное бюджетное образовательное учреждение высшего образования}\\
		\textbf{«Московский государственный технический университет имени Н. Э.~Баумана}\\
		\textbf{(национальный исследовательский университет)»}\\
		\textbf{(МГТУ им. Н. Э.~Баумана)}
	\end{minipage}

	\noindent\rule{18cm}{3pt}
	\newline\newline
	\noindent ФАКУЛЬТЕТ $\underline{\text{«Информатика и системы управления»~~~~~~~~~~~~~~~~~~~~~~~~~~~~~~~~~~~~~~~~~~~~~~~~~~~~~~~}}$ \newline\newline
	\noindent КАФЕДРА $\underline{\text{«Программное обеспечение ЭВМ и информационные технологии»~~~~~~~~~~~~~~~~~~~~~~~}}$\newline\newline\newline\newline\newline\newline\newline


	\begin{center}
		\noindent\begin{minipage}{1.3\textwidth}\centering
		\Large\textbf{   ~~~ Лабораторная работа №2}\newline
		\textbf{по дисциплине <<Анализ Алгоритмов>>}\newline\newline\newline
		\end{minipage}
	\end{center}

	\noindent\textbf{Тема} 			$\underline{\text{Алгоритмы умножения матриц}}$\newline\newline
	\noindent\textbf{Студент} 		$\underline{\text{Куликов Е. А.}}$\newline\newline
	\noindent\textbf{Группа} 		$\underline{\text{ИУ7-56Б}}$\newline\newline
	\noindent\textbf{Преподаватель} $\underline{\text{Волкова Л. Л.}}$\newline

	\begin{center}
		\vfill
		Москва~---~\the\year
		~г.
	\end{center}
	\restoregeometry
\end{titlepage}

\renewcommand{\contentsname}{СОДЕРЖАНИЕ}
\tableofcontents
\setcounter{page}{2}

% Введение
\chapter*{ВВЕДЕНИЕ}
\addcontentsline{toc}{chapter}{ВВЕДЕНИЕ}

Умножение матриц --- одна из основных операций над матрицами. Матрица, получаемая в результате операции умножения, называется произведением матриц.
Матричное умножение определено только тогда, когда число столбцов первой матрицы равно числу строк второй матрицы [1].
\parЦелью данной лабораторной работы является анализ и реализация различных алгоритмов умножения матриц, таких как классический алгоритм и алгоритм Винограда.
Для достижения поставленной цели необходимо решить следующие задачи:

\begin{itemize}
	\item[---] проанализировать классический алгоритм умножения матриц и алгоритм Винограда;
	\item[---] реализовать указанные алгоритмы;
	\item[---] провести сравнение эффективности данных алгоритмов по времени выполнения на плате STM;
        \item[---] описать и обосновать полученные результаты в отчете о выполненной лабораторной работе.
\end{itemize}


\chapter{Аналитическая часть}

В данном разделе будут рассмотрены классический алгоритм и алгоритм Винограда для умножения матриц.

\section{Классический алгоритм умножения матриц}
Первый способ умножения матриц --- классический. Пусть есть матрица $A$ размером $n \times m$ с элементами $a_{ij}$ и матрица $B$ размером $m \times k$ с элементами $b_{ij}$ (число строк второй матрицы должно быть равно числу столбцов первой матрицы). Тогда результатом умножения этих матриц будет являться матрица $C$ размером $n \times k$ с элементами:
\begin{equation}
    \label{eq1}
    c_{ij} = \sum_{k=1}^{n}a_{ik}\cdot b_{kj}
\end{equation}


\section{Алгоритм Винограда}
Алгоритм Винограда --- попытка оптимизировать классический алгоритм умножения матриц путем снижения доли <<дорогих>> операций, в частности --- умножения.
\par Классическое умножение строки на столбец:
\begin{equation}
    \label{eq2}
    c_{ij} =
\big(\begin{smallmatrix}
  u_{1} & u_{2} & u_{3} & u_{4}
\end{smallmatrix}\big)_{i}
\times
\big(\begin{smallmatrix}
  v_{1}\\v_{2}\\v_{3}\\v_{4}
\end{smallmatrix}\big)_{j} = u_{1}\cdot v_{1} + u_{2}\cdot v_{2} + u_{3}\cdot v_{3} + u_{4}\cdot v_{4}
\end{equation}
\parНа 4 элемента приходится 2 умножения.
В алгоритме Винограда формула изменена:
\begin{equation}
    \label{eq3}
    c_{ij} = (u_{1} + v_{2}) \cdot (u_{2} + v_{1}) + (u_{3} + v_{4}) \cdot (u_{4} + v_{3}) - u_{1} \cdot u_{2} - u_{3} \cdot u_{4} - v_{1} \cdot v_{2} - v_{3} \cdot v_{4}
\end{equation}
\parДля первых двух компонент со скобками получаем 1 умножение на 4 элемента. При этом третью ки четвертую компоненту можно вычислить заранее и использовать для умножения данной строки матрицы $A$ на все столбцы матрицы $B$ или данного столбца матрицы $B$ на все строки матрицы $A$.
Эти значения накапливаются в отдельных массивах --- $MulH$ и $MulV$.

\chapter{Конструкторская часть}
В данном разделе будут разработаны классический алгоритм и алгоритм Винограда умножения матриц.

\section{Модель вычислений}
Для описания трудоемкости алгоритмов вводится модель вычислений:
\begin{itemize}
	\item[---] трудоемкость базовых операций =, +, -, +=, -=, ==, !=, <, >, <=, >=, [], <<, >> принимается единичной, трудоемкость операций *, /, *=, /=, \%, \%= принимается равной 2;
	\item[---] трудоемкость оператора цикла вида $for([init];[comp];[inc])\{[body]\}$ при известных трудоемкостях его составных частей $f_{init}, f_{comp}, f_{inc}, f_{body}$ принимается за $f_{cycle} = f_{init} + f_{comp} + M \cdot (f_{body} + f_{inc} + f_{comp}$) где $M$ --- количество шагов цикла;
        \item[---] трудоемкость условного перехода принимается равной нулю, для условного оператора вида $if [condition]
            \{[code1]\}
            else
            \{[code2]\}$
        при известных трудоемкостях $f_{code1}, f_{code2}, f_{condition}$ трудоемкость оценивается как
        \begin{equation}
            f = \begin{cases}
    	f_{condition} + min(f_{code1}, f_{code2}), &\text{для лучшего случая}\\
    	f_{condition} + max(f_{code1}, f_{code2}), &\text{для худшего случая}\\
    	\end{cases}
        \end{equation}
\end{itemize}
\newpage
\section{Оценки трудоемкости алгоритмов}
\subsection{Классический алгоритм}
Трудоемкость классического алгоритма умножения матриц:
\begin{center}
\captionsetup{justification=raggedright,singlelinecheck=off}
\begin{lstlisting}[label=lst:alg_cls, language=c, numbers=none, caption=Классический алгоритм умножения матриц]
for (int i = 0; i < M; i++)
        for (int j = 0; j < Q; j++){
            C[i][j] = 0;   //3
            for (int k = 0; k < N; k++)
                C[i][j] = C[i][j] + A[i][k] * B[k][j];
        }
\end{lstlisting}
\end{center}
Трудоемкость алгоритма равна:
\begin{equation}
    f_{classical} = 14\cdot M\cdot N\cdot Q + 7\cdot Q + 4\cdot M + 2
\end{equation}

\newpage
\subsection{Алгоритм Винограда}
Трудоемкость алгоритма Винограда:
\begin{center}
\captionsetup{justification=raggedright,singlelinecheck=off}
\begin{lstlisting}[label=lst:alg_win, language=c, numbers=none, caption=Алгоритм умножения матриц Винограда]
// I - filling of MulH
for (int i = 0; i < M; i++){
    for (int k = 0; k < N / 2; k++){
        MulH[i] = MulH[i] + A[i][k * 2] * A[i][k * 2 + 1];
    }
}
// II - filling of MulV
for (int i = 0; i < Q; i++){
    for (int k = 0; k < N / 2; k++){
        MulV[i] = MulV[i] + B[k * 2][i] * B[k * 2 + 1][i];
    }
}
// III - filling of C
for (int i = 0; i < M; i++) {
    for (int j = 0; j < Q; j++){
        C[i][j] = -MulH[i] - MulV[j];
        for (int k = 0; k < N / 2; k++){
            C[i][j] = C[i][j] + (A[i][k * 2 + 1] + B[k * 2][j]) * (A[i][k * 2] + B[k * 2 + 1][j]);
        }
    }
}
// IV
if (N % 2) {
    for (int i = 0; i < M; i++){
        for (int j = 0; j < Q; j++){
            C[i][j] = C[i][j] + A[i][N - 1] * B[N - 1][j];
        }
    }
}
\end{lstlisting}
\end{center}
Общая трудоемкость первого этапа I:
\begin{equation}
    f_{I} = \frac{19}{2}\cdot M \cdot N + 6\cdot M + 2
\end{equation}
Общая трудоемкость второго этапа II:
\begin{equation}
    f_{II} = \frac{19}{2}\cdot Q \cdot N + 6\cdot Q + 2
\end{equation}
Общая трудоемкость третьего этапа III:
\begin{equation}
    f_{III} = 16\cdot M \cdot N \cdot Q + 13 \cdot M \cdot Q + 4\cdot M + 2
\end{equation}
Общая трудоемкость четвертого этапа:
\begin{equation}
	f_{IV} = \begin{cases}
	16\cdot M\cdot Q + 4\cdot M + 3, &\text{худший случай при четном N}\\
	1, &\text{лучший случай при нечетном N}\\
	\end{cases}
\end{equation}
Итоговая трудоемкость алгоритма в лучшем случае:
\begin{equation}
    f_{win} = 16\cdot M \cdot N \cdot Q + \frac{19\cdot M \cdot N}{2} + \frac{19\cdot Q \cdot N}{2} + 13\cdot M\cdot Q + 4\cdot M + 6\cdot Q + 6\cdot M + 7
\end{equation}
Итоговая трудоемкость алгоритма в худшем случае:
\begin{equation}
    f_{win} = 16\cdot M \cdot N \cdot Q + \frac{19\cdot M \cdot N}{2} + \frac{19\cdot Q \cdot N}{2} + 29\cdot M\cdot Q + 8\cdot M + 6\cdot Q + 6\cdot M + 10
\end{equation}

\newpage
\subsection{Оптимизированный алгоритм Винограда}
По варианту алгоритм был оптимизирован: ввод инкремента +=, использование двоичного сдвига влево вместо умножения на 2, введение декремента при вычислении вспомогательных массивов.
\begin{center}
\captionsetup{justification=raggedright,singlelinecheck=off}
\begin{lstlisting}[label=lst:alg_win_opt, language=c, numbers=none, caption=Оптимизированный алгоритм умножения матриц Винограда]
// I - filling of MulH
for (int i = 0; i < M; i++){
    for (int k = 0; k < N / 2; k++){
        MulH[i] -= A[i][k << 1] * A[i][k << 1 + 1];
    }
}
// II - filling of MulV
for (int i = 0; i < Q; i++){
    for (int k = 0; k < N / 2; k++){
        MulV[i] -= B[k << 1][i] * B[k << 1 + 1][i];
    }
}
// III - filling of C
for (int i = 0; i < M; i++) {
    for (int j = 0; j < Q; j++){
        C[i][j] = MulH[i] + MulV[j];
        for (int k = 0; k < N / 2; k++){
            C[i][j] += (A[i][k << 1 + 1] + B[k << 1][j]) * (A[i][k << 1] + B[k << 1 + 1][j]);
        }
    }
}
// IV
if (N % 2) {
    for (int i = 0; i < M; i++){
        for (int j = 0; j < Q; j++){
            C[i][j] += A[i][N - 1] * B[N - 1][j];
        }
    }
}
\end{lstlisting}
\end{center}
Общая трудоемкость первого этапа I:
\begin{equation}
    f_{I} = \frac{15}{2}\cdot M \cdot N + 6\cdot M + 2
\end{equation}
Общая трудоемкость второго этапа II:
\begin{equation}
    f_{II} = \frac{15}{2}\cdot Q \cdot N + 6\cdot Q + 2
\end{equation}
Общая трудоемкость третьего этапа III:
\begin{equation}
    f_{III} = \frac{27}{2}\cdot M \cdot N \cdot Q + 12 \cdot M \cdot Q + 4\cdot M + 2
\end{equation}
Общая трудоемкость четвертого этапа:
\begin{equation}
	f_{IV} = \begin{cases}
	15\cdot M\cdot Q + 4\cdot M + 3, &\text{худший случай при четном N}\\
	1, &\text{лучший случай при нечетном N}\\
	\end{cases}
\end{equation}
Итоговая трудоемкость алгоритма в лучшем случае:
\begin{equation}
    f_{win} = \frac{27\cdot M \cdot N \cdot Q}{2} + \frac{15\cdot M \cdot N}{2} + \frac{15\cdot Q \cdot N}{2} + 12\cdot M\cdot Q + 4\cdot M + 6\cdot Q + 6\cdot M + 7
\end{equation}
Итоговая трудоемкость алгоритма в худшем случае:
\begin{equation}
    f_{win} = \frac{27\cdot M \cdot N \cdot Q}{2} + \frac{15\cdot M \cdot N}{2} + \frac{15\cdot Q \cdot N}{2} + 12\cdot M\cdot Q + 19\cdot M + 6\cdot Q + 10\cdot M + 10
\end{equation}

\section{Разработка алгоритмов}

На вход каждому алгоритму подается две матрицы и их размеры --- количество строк и столбцов.

На рисунках \ref{fig:default_mul} --- \ref{fig:winograd_2} приведены стандартный алгоритм и алгоритм Винограда умножения матриц.

\clearpage

\begin{figure}[h]
	\centering
        \vspace{0pt}
	\includegraphics[scale=0.9]{img/default_mul.png}
	\caption{Стандартный алгоритм умножения матриц}
 \vspace*{3in}
	\label{fig:default_mul}
\end{figure}

\clearpage

\begin{figure}[h]
	\centering
        \vspace{0pt}
	\includegraphics[scale=0.9]{img/winograd_1.png}
	\caption{Алгоритм Винограда часть 1}
        \vspace*{3in}
	\label{fig:winograd_1}
\end{figure}
\clearpage

\begin{figure}[h]
	\centering
        \vspace{0pt}
	\includegraphics[scale=0.9]{img/winograd_2.png}
	\caption{Алгоритм Винограда часть 2}
        \vspace*{3in}
	\label{fig:winograd_2}
\end{figure}
\clearpage

\section*{Вывод}
В данном разделе были разработаны алгоритмы умножения матриц.

\clearpage

\chapter{Технологическая часть}

В данном разделе будут приведены данные о выбранном языке программирования, коды алгоритмов и тесты для каждого алгоритма.

\section{Требования к ПО}
Реализуемое ПО будет давать возможность работы в двух режимах: умножение двух введенных пользователем матриц, а также массовый замер времени умножения случайно сгенерированных квадратных матриц указанного размера.

\section{Язык программирования}
В данной работе для реализации алгоритмов был выбран язык программирования $C$ [3]. Язык предоставляет возможность работы с матрицами различных размеров, а также возможность создания пользовательского интерфейса. Также язык позволяет производить замеры процессорного времени выполнения функций с использованием функции clock [2].

\section{Реализация алгоритмов}

В листингах \ref{lst:default} --- \ref{lst:win_opt} представлены реализации алгоритма поиска полным перебором и алгоритма бинарного поиска.

\clearpage

\begin{center}
\captionsetup{justification=raggedright,singlelinecheck=off}
\begin{lstlisting}[label=lst:default, language=c, numbers=none, caption=Реализация стандартного алгоритма умножения матриц]
int** matrix_mul(int** fir, int c_row1, int c_col1, int** sec, int c_row2, int c_col2, double *process_time)
{
    int c_row = c_row1;
    int c_col = c_col2;
    if (c_col1 != c_row2)
        return NULL;
    int** result = allocate_matrix(c_row, c_col);
    if (!result)
        return NULL;
    clock_t begin = clock();
    for (int i = 0; i < c_row; i++)
        for (int j = 0; j < c_col; j++){
            result[i][j] = 0;
            for (int k = 0; k < c_col1; k++)
                result[i][j] = result[i][j] + fir[i][k] * sec[k][j];
        }
    *process_time = (double) (clock() - begin) / CLOCKS_PER_SEC;
    return result;
}
\end{lstlisting}
\end{center}

\begin{center}
\captionsetup{justification=raggedright,singlelinecheck=off}
\begin{lstlisting}[label=lst:win, language=c, numbers=none, caption=Реализация алгоритма Винограда]
int** matrix_mul_winograd(int** fir, int c_row1, int c_col1, int** sec, int c_row2, int c_col2, double *process_time)
{
    int c_row = c_row1;
    int c_col = c_col2;
    int* mulH = malloc(c_row1 * sizeof(int));
    if (!mulH)
        return NULL;
    for (int i = 0; i < c_row1; i++)
        mulH[i] = 0;
    int* mulV = malloc(c_col2 * sizeof(int));
    if (!mulV){
        free(mulH);
        return NULL;
    }
    for (int i = 0; i < c_col2; i++)
        mulV[i] = 0;
    int** result = allocate_matrix(c_row, c_col);
    if (!result) {
        free(mulH);
        free(mulV);
        return NULL;
    }
    clock_t begin = clock();
    for (int i = 0; i < c_row1; i++){
        for (int j = 0; j < c_col1 / 2; j++){
            mulH[i] = mulH[i] + fir[i][j * 2] * fir[i][j * 2 + 1];
        }
    }
    for (int i = 0; i < c_col2; i++){
        for (int j = 0; j < c_row2 / 2; j++){
            mulV[i] = mulV[i] + sec[j * 2][i] * sec[j * 2 + 1][i];
        }
    }
    for (int i = 0; i < c_row; i++) {
        for (int j = 0; j < c_col; j++){
            result[i][j] = - (mulH[i] + mulV[j]);
            for (int k = 0; k < c_col1 / 2; k++){
                result[i][j] = result[i][j] + (fir[i][k * 2 + 1]+ sec[k * 2][j]) * (fir[i][k * 2] + sec[k * 2 + 1][j]);
            }
        }
    }
    if (c_col1 % 2) {
        for (int i = 0; i < c_row; i++){
            for (int j = 0; j < c_col; j++){
                result[i][j] = result[i][j] + fir[i][c_col1 - 1] * sec[c_col1 - 1][j];
            }
        }
    }
    *process_time = (double) (clock() - begin) / CLOCKS_PER_SEC;
    free(mulH);
    free(mulV);

    return result;
}
\end{lstlisting}
\end{center}

\begin{center}
\captionsetup{justification=raggedright,singlelinecheck=off}
\begin{lstlisting}[label=lst:win_opt, language=c, numbers=none, caption=Реализация оптимизированного алгоритма винограда]
int** matrix_mul_winograd_optimized(int** fir, int c_row1, int c_col1, int** sec, int c_row2, int c_col2, double *process_time)
{
    int c_row = c_row1;
    int c_col = c_col2;
    int* mulH = malloc(c_row1 * sizeof(int));
    if (!mulH)
        return NULL;
    for (int i = 0; i < c_row1; i++)
        mulH[i] = 0;
    int* mulV = malloc(c_col2 * sizeof(int));
    if (!mulV){
        free(mulH);
        return NULL;
    }
    for (int i = 0; i < c_col2; i++)
        mulV[i] = 0;
    int** result = allocate_matrix(c_row, c_col);
    if (!result) {
        free(mulH);
        free(mulV);
        return NULL;
    }
    clock_t begin = clock();
    for (int i = 0; i < c_row1; i++){
        for (int j = 0; j < c_col1 / 2; j++){
            mulH[i] -= fir[i][j << 1] * fir[i][(j << 1) + 1];
        }
    }
    for (int i = 0; i < c_col2; i++){
        for (int j = 0; j < c_row2 / 2; j++){
            mulV[i] -= sec[j << 1][i] * sec[(j << 1) + 1][i];
        }
    }
    for (int i = 0; i < c_row; i++) {
        for (int j = 0; j < c_col; j++){
            result[i][j] = mulH[i] + mulV[j];
            for (int k = 0; k < c_col1 / 2; k++){
                result[i][j] += (fir[i][(k << 1) + 1]+ sec[k << 1][j]) * (fir[i][k << 1] + sec[(k << 1) + 1][j]);
            }
        }
    }
    if (c_col1 % 2) {
        for (int i = 0; i < c_row; i++){
            for (int j = 0; j < c_col; j++){
                result[i][j] += fir[i][c_col1 - 1] * sec[c_col1 - 1][j];
            }
        }
    }
    *process_time = (double) (clock() - begin) / CLOCKS_PER_SEC;
    free(mulH);
    free(mulV);

    return result;
}
\end{lstlisting}
\end{center}

\clearpage

\section{Функциональные тесты}
В листинге \ref{lst:test} описан вывод программы для различных функциональных тестов. Все тесты программа прошла успешно.
\begin{center}
\captionsetup{justification=raggedright,singlelinecheck=off}
\begin{lstlisting}[label=lst:test, numbers=none, caption=Функциональные тесты]
# Test 1
Please enter the number of rows and columns:
3 4
Enter the elements of the matrix:
1 2 3 4 5 6 7 8 9 10 11 12
Please enter the number of rows and columns:
4 2
Enter the elements of the matrix:
1 2 3 4 5 6 7 8
Result (default algorithm):
50 60
114 140
178 220
Result (Winograd algorithm):
50 60
114 140
178 220
Result (Optimized Winograd algorithm):
50 60
114 140
178 220

# Test 2
Please enter the number of rows and columns:
1 5
Enter the elements of the matrix:
1 2 3 4 5
Please enter the number of rows and columns:
5 1
Enter the elements of the matrix:
1 2 3 4 5
Result (default algorithm):
55
Result (Winograd algorithm):
55
Result (Optimized Winograd algorithm):
55

# Test 3
Please enter the number of rows and columns:
1 5
Enter the elements of the matrix:
1 2 3 4 5
Please enter the number of rows and columns:
5 3
Enter the elements of the matrix:
1 2 3 4 5 6 7 8 9 10 11 12 13 14 15
Result (default algorithm):
135 150 165
Result (Winograd algorithm):
135 150 165
Result (Optimized Winograd algorithm):
135 150 165
\end{lstlisting}
\end{center}

\section*{Вывод}
В данном разделе были приведены коды реализаций алгоритмов умножения матриц, а также функциональные тесты.


\newpage
\chapter{Исследовательская часть}
В данном разделе будут представлены результаты исследования эффективности работы алгоритмов по времени. Замеры проводились на плате STM32H735IGK6 c 564КБ оперативной памяти и 1МБ флэш-памяти [4].

\section{Замеры времени работы алгоритмов}
Результаты замеров времени перемножения матриц от их размеров представлены на графиках \ref{fig:even} --- \ref{fig:odd}.
Замеры производились на квадратных матрицах размерами от 30 на 30 до 100 на 100.
\begin{table}[h]
    \begin{center}
        \begin{threeparttable}
        \captionsetup{justification=raggedright,singlelinecheck=off}
        \caption{Результаты замеров времени для четных размеров матриц (ХС), с}
        \label{tbl:time_mes_even}
        \begin{tabular}{|c|c|c|c|}
            \hline
            Размер & Стандартный & Виноград & Виноград (опт) \\
            \hline
            30    & 0.002 & 0.002 & 0.001 \\
            \hline
            40    & 0.004& 0.003 & 0.003\\
            \hline
            50    & 0.007& 0.005 & 0.006\\
            \hline
            60    & 0.019&0.014 &0.012\\
            \hline
            70    & 0.022& 0.017& 0.013 \\
            \hline
            80    & 0.031& 0.025& 0.023\\
            \hline
            90    & 0.043& 0.036 & 0.032 \\
            \hline
            100    & 0.062& 0.046 & 0.041\\
            \hline
		\end{tabular}
    \end{threeparttable}
\end{center}
\end{table}


\begin{table}[h]
    \begin{center}
        \begin{threeparttable}
        \captionsetup{justification=raggedright,singlelinecheck=off}
        \caption{Результаты замеров времени для нечетных размеров матриц (ЛС), с}
        \label{tbl:time_mes_odd}
        \begin{tabular}{|c|c|c|c|}
            \hline
            Размер & Стандартный & Виноград & Виноград (опт) \\
            \hline
            31    & 0.001 & 0.002 & 0.002 \\
            \hline
            41    & 0.004& 0.004 & 0.004\\
            \hline
            51    & 0.008& 0.006 & 0.006\\
            \hline
            61    & 0.014&0.013 &0.012\\
            \hline
            71    & 0.021& 0.019& 0.017 \\
            \hline
            81    & 0.032& 0.028& 0.025\\
            \hline
            91    & 0.044& 0.039 & 0.035 \\
            \hline
            101    & 0.063& 0.052 & 0.047\\
            \hline
		\end{tabular}
    \end{threeparttable}
\end{center}
\end{table}
\newpage

\begin{figure}[h]
 \centering
 \includesvg[scale=0.7]{img/even_matr_mul.svg}
 \caption{Сравнение для четных размеров матриц}
 \label{fig:even}
\end{figure}
\newpage

\begin{figure}[h]
 \centering
 \includesvg[scale=0.7]{img/odd_matr_mul.svg}
 \caption{Сравнение для нечетных размеров матриц}
 \label{fig:odd}
\end{figure}
\newpage

\section*{Вывод}
Алгоритм Винограда является более быстрым, чем классический алгоритм умножения матриц, при дополнительных оптимизациях его можно сделать еще быстрее. Однако, алгоритм Винограда требует использования дополнительной памяти под массивы MulH и MulV. Для случая квадратных матриц вида $N \times N$ дополнительные массивы занимают долю равную $\frac{N + N}{2\cdot N^2 + 2\cdot N} = \frac{1}{N + 1}$ от общей памяти, используемой для вычислений (не считая матрицы-результата).


\chapter*{ЗАКЛЮЧЕНИЕ}
\addcontentsline{toc}{chapter}{ЗАКЛЮЧЕНИЕ}

В ходе выполнения данной лабораторной работы были проанализированы классический алгоритм и алгоритм Винограда умножения матриц, а также решены следующие задачи:
\begin{itemize}
	\item[---] проанализирован классический алгоритм умножения матриц и алгоритм Винограда;
	\item[---] реализованы указанные алгоритмы;
	\item[---] проведено сравнение эффективности данных алгоритмов по времени выполнения на плате STM;
        \item[---] описаны и обоснованы полученные результаты в отчете о выполненной лабораторной работе.
\end{itemize}

Алгоритм Винограда является более быстрым, чем классический алгоритм умножения матриц, при дополнительных оптимизациях его можно сделать еще быстрее. Однако, алгоритм Винограда требует использования дополнительной памяти под массивы MulH и MulV. Для случая квадратных матриц вида $N \times N$ дополнительные массивы занимают долю равную $\frac{N + N}{2\cdot N^2 + 2\cdot N} = \frac{1}{N + 1}$ от общей памяти, используемой для вычислений (не считая матрицы-результата).

\vspace{5mm}

\renewcommand{\bibname}{СПИСОК ИСПОЛЬЗОВАННЫХ ИСТОЧНИКОВ}
\begin{thebibliography}{5}
	\bibitem{bib1}
	Линейная алгебра, учебное пособие [Электронный ресурс]. Режим доступа: \url{https://elar.urfu.ru/bitstream/10995/78551/1/978-5-7996-2776-8_2019.pdf?ysclid=m29hw3nfrq373542650} (дата обращения: 14.10.2024).
        \bibitem{bib2}
	Функция clock() [Электронный ресурс]. Режим доступа: \url{https://en.cppreference.com/w/c/chrono/clock} (дата обращения: 14.10.2024).
        \bibitem{bib3}
	Стандарт языка Си [Электронный ресурс]. Режим доступа: \url{https://cs.petrsu.ru/~akolosov/inf/2009/docs/c99__iso-9899-tc2.pdf} (дата обращения: 14.10.2024).
        \bibitem{bib3}
	Discovery kit with STM32H735IG MCU [Электронный ресурс]. Режим доступа: \url{https://www.st.com/en/evaluation-tools/stm32h735g-dk.html} (дата обращения: 14.10.2024).
\end{thebibliography}

\addcontentsline{toc}{chapter}{СПИСОК ИСПОЛЬЗОВАННЫХ ИСТОЧНИКОВ}

\end{document}

