\documentclass[a4paper,14pt, unknownkeysallowed]{extreport}

\usepackage{cmap} % Улучшенный поиск русских слов в полученном pdf-файле
\usepackage[T2A]{fontenc} % Поддержка русских букв
\usepackage[utf8]{inputenc} % Кодировка utf8
\usepackage[english,russian]{babel} % Языки: русский, английский
\usepackage{enumitem}
\usepackage{svg}
%\usepackage{titlesec}

%\titleformat{\section}{\filcenter\normalfont\Large\bfseries}{\thesection.}{0.2em}


\usepackage{threeparttable}

\usepackage[14pt]{extsizes}

\usepackage{caption}
\captionsetup{labelsep=endash}
\captionsetup[figure]{name={Рисунок}}

% \usepackage{ctable}
% \captionsetup[table]{justification=raggedleft,singlelinecheck=off}

\usepackage{amsmath}

\usepackage{geometry}
\geometry{left=30mm}
\geometry{right=10mm}
\geometry{top=20mm}
\geometry{bottom=20mm}

\usepackage{titlesec}
\titleformat{\section}
	{\normalsize\bfseries}
	{\thesection}
	{1em}{}
\titlespacing*{\chapter}{0pt}{-30pt}{8pt}
\titlespacing*{\section}{\parindent}{*4}{*4}
\titlespacing*{\subsection}{\parindent}{*4}{*4}

\usepackage{setspace}
\onehalfspacing % Полуторный интервал

\frenchspacing
\usepackage{indentfirst} % Красная строка

\usepackage{titlesec}
\titleformat{\chapter}{\LARGE\bfseries}{\thechapter}{20pt}{\LARGE\bfseries}
\titleformat{\section}{\Large\bfseries}{\thesection}{20pt}{\Large\bfseries}

\usepackage{multirow}
\usepackage{listings}
\usepackage{xcolor}

\lstset{%
	language=python,
	basicstyle=\small\sffamily,
	numbers=left,
	stepnumber=1,
	numbersep=5pt,
	frame=single,
	tabsize=4,
	captionpos=t,
	breaklines=true,					
	breakatwhitespace=true,
	escapeinside={\#*}{*)},
	backgroundcolor=\color{white},
}


\usepackage{pgfplots}
\usetikzlibrary{datavisualization}
\usetikzlibrary{datavisualization.formats.functions}

\usepackage{graphicx}
\newcommand{\img}[3] {
	\begin{figure}[h!]
		\center{\includegraphics[height=#1]{img/#2}}
		\caption{#3}
		\label{img:#2}
	\end{figure}
}


\usepackage[justification=centering]{caption} % Настройка подписей float объектов

\usepackage[unicode,pdftex]{hyperref} % Ссылки в pdf
\hypersetup{hidelinks}

\usepackage{csvsimple}

\newcommand{\code}[1]{\texttt{#1}}





\begin{document}

% Титульная страница
\begin{titlepage}
	\newgeometry{pdftex, left=2cm, right=2cm, top=2.5cm, bottom=2.5cm}
	\fontsize{12pt}{12pt}\selectfont
	\noindent \begin{minipage}{0.15\textwidth}
		\includegraphics[width=\linewidth]{img/b_logo.jpg}
	\end{minipage}
	\noindent\begin{minipage}{0.9\textwidth}\centering
		\textbf{Министерство науки и высшего образования Российской Федерации}\\
		\textbf{Федеральное государственное бюджетное образовательное учреждение высшего образования}\\
		\textbf{«Московский государственный технический университет имени Н. Э.~Баумана}\\
		\textbf{(национальный исследовательский университет)»}\\
		\textbf{(МГТУ им. Н. Э.~Баумана)}
	\end{minipage}
	
	\noindent\rule{18cm}{3pt}
	\newline\newline
	\noindent ФАКУЛЬТЕТ $\underline{\text{«Информатика и системы управления»~~~~~~~~~~~~~~~~~~~~~~~~~~~~~~~~~~~~~~~~~~~~~~~~~~~~~~~}}$ \newline\newline
	\noindent КАФЕДРА $\underline{\text{«Программное обеспечение ЭВМ и информационные технологии»~~~~~~~~~~~~~~~~~~~~~~~}}$\newline\newline\newline\newline\newline\newline\newline
	
	
	\begin{center}
		\noindent\begin{minipage}{1.3\textwidth}\centering
		\Large\textbf{   ~~~ Лабораторная работа №1}\newline
		\textbf{по дисциплине <<Анализ Алгоритмов>>}\newline\newline\newline
		\end{minipage}
	\end{center}
	
	\noindent\textbf{Тема} 			$\underline{\text{Расстояние Левенштейна и Дамерау--Левенштейна}}$\newline\newline
	\noindent\textbf{Студент} 		$\underline{\text{Куликов Е. А.}}$\newline\newline
	\noindent\textbf{Группа} 		$\underline{\text{ИУ7-56Б}}$\newline\newline
	\noindent\textbf{Преподаватель} $\underline{\text{Волкова Л. Л.}}$\newline
	
	\begin{center}
		\vfill
		Москва~---~\the\year
		~г.
	\end{center}
	\restoregeometry
\end{titlepage}


% Страница содержания
\renewcommand{\contentsname}{Содержание} 
\tableofcontents
\setcounter{page}{2}




% Введение
\chapter*{Введение}
\addcontentsline{toc}{chapter}{Введение}

В данной рабораторной работе изучается расстояние Левенштейна.\newline Расстояние Левенштейна --- минимальное количество операций (вставки, удаления, замены), которое необходимо для перевода одной строки в другую. Это расстояние помогает определить, насколько <<близки>> две  строки.

Расстояние Левенштейна применяется в теории информации, компьютерной лингвистике и биоинформатике для:

\begin{itemize}
	\item[---] автозамены(исправления опечаток) и автозаполнения;
	\item[---] сравнения текстовых файлов утилитой diff, например для определения различий между двумя файлами в Linux или разницы между двумя Git-деревьями;
	\item[---] сравнения цепочек ДНК.
\end{itemize}

Целью данной лабораторной работы является изучение и реализация алгоритмов определения расстояния Левенштейна (итерационного, рекурсивного и рекурсивного с кэшированием) и Дамерау--Левенштейна(только итерационного). Для достижения поставленной цели необходимо решить следующие задачи:

\begin{itemize}
	\item[---] изучить итеративный и рекурсивный методы вычисления расстояния Левенштейна и Дамерау--Левенштейна;
	\item[---] реализовать указанные алгоритмы поиска расстояния Левенштейна и Дамерау--Левенштейна;
	\item[---] провести сравнение эффективности указанных алгоритмов по затраченному процессорному времени и по затраченной памяти;
	\item[---] описать и обосновать полученные результаты в отчете о выполненной лабораторной работе.
\end{itemize}





\chapter{Аналитическая часть}

В данном разделе будут рассмотрены алгоритмы нахождения расстояний Левенштейна и Дамерау--Левенштейна.

\section{Расстояние Левенштейна}

Расстояние Левенштейна [1] --- это минимальное количество операций вставки, удаления и замены, необходимых для превращения одной строки в другую.

Указанные операции имеют цену (штраф)[4], при этом в общем случае цены операций могут не совпадать. В этой работе будут рассматриваться алгоритмы, в которых цены редакторских операций одинаковы и равны 1.
В общем случае ($\lambda$ означает пустую строку).

\begin{itemize}
	\item[---] $w(a, b)$ --- цена замены символа $a$ на $b$, R (от англ. replace);
	\item[---] $w(\lambda, b)$ --- цена вставки символа $b$, I (от англ. insert);
	\item[---] $w(a, \lambda)$ --- цена удаления символа $a$, D (от англ. delete).
\end{itemize}

\section{Вычисление Расстояния Левенштейна}
Для решения задачи о расстоянии Левенштейна находится последовательность операций, минимизирующая суммарную цену. В данной лабораторной работе рассматривается частный случай поиска этого расстояния при:

\begin{itemize}
	\item[---] $w(a, a) = 0$;
	\item[---] $w(a, b) = 1$, $a \neq b$;
	\item[---] $w(\lambda, b) = 1$;
	\item[---] $w(a, \lambda) = 1$.
\end{itemize}

Имеется две строки $S_{1}$ и $S_{2}$, длинной $M$ и $N$ соответственно. Расстояние Левенштейна рассчитывается по рекуррентной формуле (\ref{eq:L}):

$D(S_{1}, S_{2}) = D(M, N)$, где

\begin{equation}
	\label{eq:L}
	D(i, j) = \begin{cases}
	0, &\text{i = 0, j = 0}\\
	i, &\text{j = 0, i > 0}\\
	j, &\text{i = 0, j > 0}\\
	min \begin{cases}
		D(i, j - 1) + 1\\
		D(i - 1, j) + 1\\
		D(i - 1, j - 1) + m(S_{1}[i], S_{2}[i]) \\
	\end{cases}
	&\text{i > 0, j > 0}
	\end{cases}
\end{equation}

\vspace{5mm}

где функция $m(a, b)$ определена как:
\begin{equation}
\label{eq:m}
m(a, b) = \begin{cases}
0 &\text{если a = b,}\\
1 &\text{иначе}
\end{cases}.
\end{equation}


Рекурсивный алгоритм реализует формулу (\ref{eq:L}). Функция $D$ составлена таким образом, что для перевода из строки $a$ в строку $b$ требуется выполнить последовательно несколько операций (удаление, вставка, замена) в некоторой последовательности. Полагая, что $a'$, $b'$ — строки $a$ и $b$ без последнего символа соответственно, цена преобразования из строки $a$ в строку $b$ может быть выражена как:

\begin{itemize}
	\item[---] сумма цены преобразования строки $a'$ в $b'$ и цены проведения операции удаления, которая необходима для преобразования $a'$ в $a$;
	\item[---] сумма цены преобразования строки $a'$ в $b'$ и цены проведения операции вставки, которая необходима для преобразования $b'$ в $b$;
	\item[---] сумма цены преобразования из $a'$ в $b'$ и операции замены, предполагая, что $a$ и $b$ оканчиваются на разные символы;
	\item[---] цена преобразования из $a'$ в $b'$, предполагая, что $a$ и $b$ оканчиваются на один и тот же символ.
\end{itemize}

Ценой преобразования будет минимальное значение приведенных вариантов.

\section{Расстояние Дамерау--Левенштейна}

Расстояние Дамерау--Левенштейна --- минимальное число операций вставки, удаления, замены и транспозиции соседних символов. То есть, по сравнению с расстоянием Левенштейна добавляется еще одна редакторская операция — транспозиция T (от англ. transposition).

\section{Вычисление расстояния \newline Дамерау--Левенштейна}

Расстояние Дамерау--Левенштейна может быть вычисленно по рекуррентной формуле (\ref{eq:DL}):

\begin{equation}
	\label{eq:DL}
	D(i, j) = \begin{cases}
	0 \qquad\qquad\qquad\qquad\qquad\qquad\qquad ,j = 0, i = 0\\
	i \qquad\qquad\qquad\qquad\qquad\qquad\qquad ,j = 0, i > 0\\
	j \qquad\qquad\qquad\qquad\qquad\qquad\qquad ,j > 0, i = 0\\
	min \begin{cases}
		D(i, j - 1) + 1\\
		D(i - 1, j) + 1\\
		D(i - 1, j - 1) + \\\qquad\qquad m(S_{1}[i], S_{2}[i])\\
		D(i - 2, j - 2) + \\\qquad\qquad m(S_{1}[i], S_{2}[i]) \\
	\end{cases}
	\begin{aligned}
		& , \text{если i > 1, j > 1} \\
		& , S_{1}[i] = S_{2}[j - 1]  \\
		& , S_{1}[j] =  S_{2}[i - 1] \\
	\end{aligned}\\
	min \begin{cases}
		D(i, j - 1) + 1\\
		D(i - 1, j) + 1 &  \text{,иначе}\\
		D(i - 1, j - 1) + \\\qquad\qquad m(S_{1}[i], S_{2}[i]) \\
	\end{cases}
	\end{cases}
\end{equation}
	    
\section*{Вывод}
В данном разделе были рассмотрены расстояния Левенштейна и Дамерау--Левенштейна. Формулы для нахождения этих расстояний, а следовательно, алгоритмы могут быть реализованы рекурсивно и итерационно.


\chapter{Конструкторская часть}
В данном разделе будут разработаны алгоритмы нахождения расстояний Левенштейна и Дамерау--Левенштейна.

\section{Разработка алгоритмов}

На вход алгоритмов падаются строки s1 и s2, которые могут содержать как русские, так и английские буквы, на выходе получаем едиственное число -- искомое расстояние.

На рис. \ref{fig:L_iter} --- \ref{fig:DL_iter} приведены схемы рекурсивных и итерационных алгоритмов Левенштейна и Дамерау--Левенштейна.

\clearpage

\begin{figure}[h]
	\centering
	\includegraphics[scale=0.6]{img/iterative_lev.png}
	\caption{Итеративный алгоритма нахождения расстояния Левенштейна}
	\label{fig:L_iter}
\end{figure}

\clearpage

\begin{figure}[h]
	\centering
	\includegraphics[scale=0.6]{img/recursive_lev.png}
	\caption{Рекурсивный алгоритм нахождения расстояния Левенштейна}
	\label{fig:L_recur}
\end{figure}

\clearpage

\begin{figure}[h]
	\centering
	\includegraphics[scale=0.55]{img/recursive_cash_lev.png}
	\caption{Рекурсивный алгоритм нахождения расстояния Левенштейна с кэшированием}
	\label{fig:L_rec_cash}
\end{figure}

\clearpage

\begin{figure}[h]
	\centering
	\includegraphics[scale=0.48]{img/iterative_dam_lev.png}
	\caption{Итеративный алгоритм нахождения расстояния Дамерау--Левенштейна}
	\label{fig:DL_iter}
\end{figure}

\section*{Вывод}
В данном разделе были разработаны алгоритмы для итерационного, рекурсивного, рекурсивного с мемоизацией алгоритмов поиска расстояния Левенштейна, а также итерационного алгоритма поиска расстояния Дамерау-Левенштейна.

\clearpage

\chapter{Технологическая часть}

В данном разделе будут приведены данные о выбранном языке программирования, коды алгоритмов и тесты для каждого алгоритма.

\section{Требования к ПО}
Реализуемое ПО будет давать возможность выбрать алгоритм, ввести две сравниваемые строки и вывести результат вычислений, а также опционально вывести матрицу расстояний для итерационных алгоритмов. Должна быть реализована возможность сравнения эффективности алгоритмов по времени выполнения.

\section{Язык программирования}
В данной работе для реализации алгоритмов был выбран язык программирования $Python$ [2]. Язык предоставляет возможности для оценки затрат времени и памяти алгоритма. Время работы было замерено с помощью функции \textit{process\_time\_ns} из библиотеки $time$ [3].


\section{Реализация алгоритмов}

В листингах \ref{lst:lev_iterative} --- \ref{lst:dam_lev_iterative} представлены реализации алгоритмов нахождения расстояния Левенштейна и Дамерау–Левенштейна.

\clearpage

\begin{center}
\captionsetup{justification=raggedright,singlelinecheck=off}
\begin{lstlisting}[label=lst:lev_iterative,caption=Реализация итерационного нахождения расстояния Левенштейна]
def levenstein_iterative(fir_word: str, sec_word: str) -> (int, list):
    fir_len = len(fir_word)
    sec_len = len(sec_word)

    matrice = [[0] * (sec_len + 1) for _ in range(fir_len + 1)]

    for i in range(fir_len + 1):
        matrice[i][0] = i
    for i in range(sec_len + 1):
        matrice[0][i] = i

    for i in range(1, fir_len + 1):
        for j in range(1, sec_len + 1):
            insert = matrice[i][j - 1] + insert_cost
            delete = matrice[i - 1][j] + delete_cost
            change = matrice[i - 1][j - 1]
            if fir_word[i - 1] != sec_word[j - 1]:
                change += change_cost

            matrice[i][j] = min(insert, delete, change)

    return matrice[fir_len][sec_len], matrice
\end{lstlisting}
\end{center}

\clearpage

\begin{center}
\captionsetup{justification=raggedright,singlelinecheck=off}
\begin{lstlisting}[label=lst:lev_recursive,caption=Реализация нахождения расстояния Левенштейна рекурсивно]
def levenstein_recursive(fir_word: str, sec_word: str) -> int:
    fir_len = len(fir_word)
    sec_len = len(sec_word)

    def recursion(fir_word: str, fir_len: int, sec_word: str, sec_len: int) -> int:
        if fir_len == 0 or sec_len == 0:
            return max(fir_len, sec_len)
        change = recursion(fir_word, fir_len - 1, sec_word, sec_len - 1)
        if fir_word[fir_len - 1] != sec_word[sec_len - 1]:
            change += change_cost

        return min(insert, delete, change)

    return recursion(fir_word, fir_len, sec_word, sec_len)
\end{lstlisting}
\end{center}

\clearpage

\begin{center}
\captionsetup{justification=raggedright,singlelinecheck=off}
\begin{lstlisting}[label=lst:lev_recursive_hash,caption=Реализация нахождения расстояния Левенштейна рекурсивно с кэшированием]
def levenstein_recursive_memoization(fir_word: str, sec_word: str) -> int:
    fir_len = len(fir_word)
    sec_len = len(sec_word)
    answers = {}

    def recursion_memoization(fir_word: str, fir_len: int, sec_word: str, sec_len: int) -> int:
        if fir_len == 0 or sec_len == 0:
            return max(fir_len, sec_len)

        key = f"{fir_len} {sec_len}"
        if key in answers:
            return answers[key]
        change = recursion_memoization(fir_word, fir_len - 1, sec_word, sec_len - 1)
        if fir_word[fir_len - 1] != sec_word[sec_len - 1]:
            change += change_cost

        result = min(insert, delete, change)
        answers[key] = result
        return result

    return recursion_memoization(fir_word, fir_len, sec_word, sec_len)
\end{lstlisting}
\end{center}

\clearpage

\begin{center}
\captionsetup{justification=raggedright,singlelinecheck=off}
\begin{lstlisting}[label=lst:dam_lev_iterative,caption=Реализация итерационного нахождения расстояния Дамерау–Левенштейна]
def damerau_levenstein_iterative(fir_word: str, sec_word: str) -> (int, list):
    fir_len = len(fir_word)
    sec_len = len(sec_word)

    matrice = [[0] * (sec_len + 1) for _ in range(fir_len + 1)]

    for i in range(fir_len + 1):
        matrice[i][0] = i
    for i in range(sec_len + 1):
        matrice[0][i] = i

    for i in range(1, fir_len + 1):
        for j in range(1, sec_len + 1):
            change = matrice[i - 1][j - 1]
            if fir_word[i - 1] != sec_word[j - 1]:
                change += change_cost
            swap = insert + delete + change
            if i >= 2 and j >= 2:
                if fir_word[i - 1] == sec_word[j - 2] and fir_word[i - 2] == sec_word[j - 1]:
                    swap = matrice[i - 2][j - 2] + swap_cost

            matrice[i][j] = min(insert, delete, change, swap)

    return matrice[fir_len][sec_len], matrice
\end{lstlisting}
\end{center}

\clearpage

\section{Функциональные тесты}

В таблице \ref{tbl:functional_test} приведены функциональные тесты для алгоритмов вычисления расстояния Левенштейна и Дамерау—Левенштейна. Числа в скобках i(j) обозначают несовпадение ожидаемых результатов вычисления расстояния Левенштейна и Дамерау--Левенштейна. Все тесты программа прошла успешно.


\begin{table}[h]
	\begin{center}
	\begin{threeparttable}
		\captionsetup{justification=raggedright,singlelinecheck=off}
		\caption{\label{tbl:functional_test} Функциональные тесты}
\begin{tabular}{|l|l|lllll|}


\hline
\multirow{2}{*}{1-я строка} & \multirow{2}{*}{2-я строка} & \multicolumn{5}{l|}{Результаты}                                                                                                       \\ \cline{3-7} 
                            &                             & \multicolumn{1}{l|}{Эталон} & \multicolumn{1}{l|}{И. Лев.} & \multicolumn{1}{l|}{Р. Лев.} & \multicolumn{1}{l|}{Р. х. Лев.} & И. Д-Л \\ \hline
{[}{]}                      & {[}{]}                      & \multicolumn{1}{l|}{0}      & \multicolumn{1}{l|}{0}       & \multicolumn{1}{l|}{0}       & \multicolumn{1}{l|}{0}            & 0     \\ \hline
{[}{]}                      & Moscow                      & \multicolumn{1}{l|}{6}      & \multicolumn{1}{l|}{6}       & \multicolumn{1}{l|}{6}       & \multicolumn{1}{l|}{6}            & 6     \\ \hline
Moscow                      & cow                         & \multicolumn{1}{l|}{3}      & \multicolumn{1}{l|}{3}       & \multicolumn{1}{l|}{3}       & \multicolumn{1}{l|}{3}            & 3     \\ \hline
Moscow                      & Mocsow                      & \multicolumn{1}{l|}{2(1)}   & \multicolumn{1}{l|}{2}       & \multicolumn{1}{l|}{2}       & \multicolumn{1}{l|}{2}            & 1     \\ \hline
Moscow                      & Mosco                       & \multicolumn{1}{l|}{1}      & \multicolumn{1}{l|}{1}       & \multicolumn{1}{l|}{1}       & \multicolumn{1}{l|}{1}            & 1     \\ \hline
Moscow                      & Moscoe                      & \multicolumn{1}{l|}{1}      & \multicolumn{1}{l|}{1}       & \multicolumn{1}{l|}{1}       & \multicolumn{1}{l|}{1}            & 1     \\ \hline
Moscow                      & wocsoM                      & \multicolumn{1}{l|}{4(3)}   & \multicolumn{1}{l|}{4}       & \multicolumn{1}{l|}{4}       & \multicolumn{1}{l|}{4}            & 3     \\ \hline
Moscow                      & Russia                      & \multicolumn{1}{l|}{5}      & \multicolumn{1}{l|}{5}       & \multicolumn{1}{l|}{5}       & \multicolumn{1}{l|}{5}            & 5     \\ \hline
\end{tabular}
\end{threeparttable}
\end{center}
\end{table}

\section*{Вывод}
В данном разделе были приведены коды реализаций итерационных алгоритмов поиска расстояний Левенштейна и Дамерау-Левенштейна, рекурсивных алгоритмов поиска расстояния Левенштейна, а также функциональные тесты.



\chapter{Исследовательская часть}
В данном разделе будут представлены результаты исследования эффективности работы алгоритмов по времени работы.

\section{Технические характеристики}

В данном разделе представлены технические характеристики устройства, на котором проводилось исследование.

\begin{itemize}
    \item[---] Операционная система: Windows11 Домашняя, версия 23Н2, сборка ОС 22631.4037;
    \item[---] Оперативная память: 16 ГБ;
    \item[---] Процессор: 13th Gen Intel(R) Core(TM) i5-13500H   2.60 ГГц.
\end{itemize}

При тестировании ноутбук был включен в сеть электропитания и нагружен только встроенными приложениями окружения, а также системой тестирования.

\section{Время выполнения алгоритмов}

Результаты замеров времени работы алгоритмов нахождения расстояний Левенштейна и Дамерау–Левенштейна приведены в таблице \ref{tbl:time_measurements}. Замеры времени проводились на строках одинаковой длины, состоящих из строчных латинских символов, их результаты усреднялись для 1000 одинаковых измерений.

\begin{table}[h]
	\begin{center}
		\begin{threeparttable}
		\captionsetup{justification=raggedright,singlelinecheck=off}
		\caption{Время работы алгоритмов (в микросекундах)}
		\label{tbl:time_measurements}
		\begin{tabular}{|c|c|c|c|c|}
			\hline
			Длина строк &  Лев итер.  & Лев рек. & Лев рек. кэш & Дам-Лев итер. \\
                \hline
			0    & 0.568 &   0.435 &    1.04 &   0.703 \\
			\hline
			1    & 0.952 &   0.837 &    1.55 &   0.954 \\ 
			\hline
			2    & 1.68 &     2.5 &    3.56 &    2.14 \\ 
			\hline
			3    & 2.81 &    11.7 &    7.42 &    3.82 \\ 
			\hline
			4    & 4.11 &    59.2 &    11.6 &    5.03 \\ 
			\hline
			5    & 5.74 & 2.97$\cdot 10^{2}$ &    18.4 &    7.35 \\ 
			\hline
			6    & 7.94 & 1.59$\cdot 10^{3}$ &    31.1 &    11.0 \\ 
			\hline
			7    & 11.7 & 8.64$\cdot 10^{3}$ &    39.8 &    13.1 \\ 
			\hline
			8    & 13.1 & 4.7$\cdot 10^{4}$ &    48.4 &    16.9 \\ 
                \hline
                9    & 15.9 & 2.64$\cdot 10^{5}$ &    61.2 &    21.1 \\
                \hline
                10    & 19.6 & 1.44$\cdot 10^{6}$ &    80.1 &    26.3 \\
   
			\hline
		\end{tabular}
		\end{threeparttable}
    \end{center}
\end{table}


\begin{figure}[h]
	\centering
	\includegraphics[scale=1]{img/Figure_1.png}
	\caption{Сравнение итеративных алгоритмов вычисления расстояний Левенштейна и Дамерау--Левенштейна}
	\label{fig:DL_table}
\end{figure}

\begin{figure}[h]
	\centering
	\includegraphics[scale=1]{img/Figure_2.png}
	\caption{Сравнение рекурсивных алгоритмов вычисления расстояния Левенштейна}
	\label{fig:DL_table}
\end{figure}

\begin{figure}[h]
	\centering
	\includegraphics[scale=1]{img/Figure_3.png}
	\caption{Сравнение итеративного и рекурсивного(с кэшем) алгоритмов вычисления расстояния Левенштейна}
	\label{fig:DL_table}
\end{figure}

\clearpage

\section{Эффективность алгоритмов по памяти}

Алгоритмы Левенштейна и Дамерау--Левенштейна не отличаются по использованию памяти, так как используют матрицы одинаковых размеров. Можно сравнить рекурсивную и итерационную реализацию алгоритмов для поиска расстояния Левенштейна.

Пусть $S_{1}, S_{2}$ --- строки, size --- функция, возвращающая размер аргумента; len --- функция, возвращающая длину строки, char --- символьный тип, int --- целочисленный.

Для итеративной реализации расход памяти рассчитывается следующим образом: память затрачивается на хранение матрицы из $(len(S_{1}) + 1) \cdot (len(S_{2}) + 1)$ чисел типа $int$ --- расстояний Левенштейна для всех возможных подстрок $S_{1}, S_{2}$, на хранение самих строк --- двух массивов типа $char$ с длинами\\ $len(S_{1}), len(S_{2})$ и на хранение длин строк и ответа --- еще три $int$-а.
Итого:
\begin{equation}
	(len(S_{1}) + 1) \cdot (len(S_{2}) + 1) \cdot size(int)  + (len(S_{1}) + len(S_{2})) \cdot (size(char) + 3 \cdot size(int)
\end{equation}

При рекурсивной реализации память затрачивается на сами рекурсивные вызовы и на хранение двух входных строк.
Максимальная глубина стека вызовов при рекурсивной реализации равна сумме длин входящих строк, а на каждый вызов функции требуется еще 3 дополнительных переменных типа $int$ (длины строк и ответ), соответственно, максимальный расход памяти

\begin{equation}
	(len(S_{1}) + len(S_{2})) \cdot ((len(S_{1}) + len(S_{2})) \cdot size(char) + 3 \cdot Size(\text{int}))
\end{equation}

\noindent

По затратам памяти итеративные алгоритмы проигрывают рекурсивным, так как они хранят одновременно все расстояния Левенштейна для всех подстрок, максимальный размер используемой памяти в них растёт как произведение длин строк, в то время как у рекурсивного алгоритма — как сумма длин строк.

\clearpage

\section*{Вывод}

Итеративные алгоритмы для поиска расстояний Левенштейна и Дамерау--Левенштейна не различаются по количеству используемой памяти, однако алгоритм для поиска расстояния Дамерау--Левенштейна работает немного дольше (см. рисунок 4.1), так как он является дополненной версией предыдущего алгоритма (на один if больше, а также проверки двух последних символов строк на возможность перестановки).
Итеративные алгоритмы работают существенно быстрее чем рекурсивные (см. рисунок 4.3). При этом рекурсивный алгоритм с кэшированием работает гораздо быстрее простого рекурсивного алгоритма (см. рисунок 4.2), так как в алгоритме с кэшированием большое количество рекурсивных вызовов, вычисляющих одни и те же данные заменяются на обращение к кэшу с уже вывчисленными значениями. Уже на длине строк в 10 символов алгоритм с кэшированием оказывается быстрее простого рекурсивного алгоритма в 18406 раз. Однако алгоритм с кэшированием требует дополнительной памяти в виде словаря-кэша.


\chapter*{Заключение}
\addcontentsline{toc}{chapter}{Заключение}

В ходе выполнения данной лабораторной работы были изучены итеративные и рекурсивные алгоритмы поиска расстояний Левенштейна и Дамерау--Левенштейна, а также решены следующие задачи:
\begin{itemize}
	\item[---] изучены итеративный и рекурсивный методы вычисления расстояния Левенштейна и Дамерау--Левенштейна;
	\item[---] реализованы указанные алгоритмы поиска расстояния Левенштейна и Дамерау--Левенштейна;
	\item[---] проведено сравнение эффективности указанных алгоритмов по затраченному процессорному времени и по затраченной памяти;
	\item[---] описаны и обоснованы полученные результаты в отчете о выполненной лабораторной работе.
\end{itemize}

Итеративные алгоритмы для поиска расстояний Левенштейна и Дамерау--Левенштейна не различаются по количеству используемой памяти, однако алгоритм для поиска расстояния Дамерау--Левенштейна работает немного дольше (см. рисунок 4.1, график 1), так как он является дополненной версией предыдущего алгоритма (на один условный оператор больше, а также проверки двух последних символов строк на возможность перестановки).
Итеративные алгоритмы работают существенно быстрее чем рекурсивные (см. рисунок 4.1, график 3). При этом рекурсивный алгоритм с кэшированием работает гораздо быстрее простого рекурсивного алгоритма (см. рисунок 4.1, график 2), так как в алгоритме с кэшированием большое количество рекурсивных вызовов, вычисляющих одни и те же данные заменяются на обращение к хэшу с уже вывчисленными значениями. Уже на длине строк в 10 символов алгоритм с кэшированием оказывается быстрее простого рекурсивного алгоритма в 18406 раз. Однако алгоритм с кэшированием требует дополнительной памяти в виде словаря-кэша.
Поставленная цель достигнута, все задачи решены.

\vspace{5mm}



\renewcommand{\bibname}{Список использованных источников}
\begin{thebibliography}{5}
	\bibitem{bib1}
	Левенштейн В. И. Двоичные коды с исправлением выпадений, вставок и замещений символов. –- М.: Доклады АН СССР, 1965. Т. 163. С. 845– 848.
	\bibitem{bib2}
	Документация языка Welcome to Python [Электронный ресурс]. Режим доступа: \url{https://www.python.org} (дата обращения: 12.09.2024).
	\bibitem{bib3}
	Документация модуля time —- Time access and conversions [Электронный ресурс]. Режим доступа: \url{https://docs.python.org/3/library/time.html#time.process_time_ns} (дата обращения: 12.09.2024).
	\bibitem{bib5}
	Кормановский, М.В. Граф на основе расстояния Левенштейна и его визуализация / М. В. Кормановский, Н. П. Артюхин, А. А. Косарев [и др.] // Проблемы лингвистики и лингводидактики в неязыковом вузе : 5-я Международная научно-практическая конференция: сборник материалов конференции. В 2-х томах, Москва, 15 декабря 2022 года. Том 1. – Москва: Московский государственный технический университет имени Н.Э. Баумана (национальный исследовательский университет), 2023. – С. 310-319.
\end{thebibliography}

\addcontentsline{toc}{chapter}{Список использованных источников}

\end{document}
