% Содержимое отчета по курсу Анализ алгоритмов

\aaunnumberedsection{ВВЕДЕНИЕ}{sec:intro}

Цель работы --- получение навыка организации параллельных вычислений на основе нативных потоков.

Задачи работы:
\begin{itemize}
    \item анализ предметной области;
    \item разработка алгоритма обработки данных;
    \item создание ПО, реализующего разработанный алгоритм;
    \item исследование характеристик созданного ПО.
\end{itemize}

\aasection{Входные и выходные данные}{sec:input-output}

Входными данными программы является ссылка на сайт с кулинарными рецептами, в данном варианте --- на сайт \url{https://www.kuhnyatv.ru/recipes}, название папки, в которую нужно сохранять рецепты, количество рецептов, которые нужно сохранить и количество потоков которые нужно создать. Выходными данными программы является папка с заданным именем, внутри которой располагается заданное количество рецептов в сормате html--страниц.

\aasection{Преобразование входных данных в выходные}{sec:algorithm}

Программа считывает указанные входные данные и в зависимости от указанного количества потоков выполняет поиск страниц с рецептами в последовательном режиме (при количестве потоков равном 0) или параллельном режиме, создавая указанное количество потоков.
\parКаждый поток (в последовательном режиме --- главный поток, в параллельном --- каждый из созданных потоков) выполняет поиск первой необработанной ссылки в общем для всех потоков множестве, помечает эту ссылку обработанной, обращается к сайту по этой ссылке (для этого используется библиотека cpr ~\cite{cpr_lib}), получая html--страницу. На этой странице производится поиск всех ссылок (с помощью регулярных выражений библиотеки regex ~\cite{regex_lib}), каждая из которых, если уже не находится в общем множестве, помещается в него и помечается необработанной. Далее проверяется, является ли страница рецептом с помощью поиска на странице ключевых слов --- "Продукты" и "Способ приготовления". Если страница является рецептом, ее исходный текст помещается в html--файл в указанную во входных данных папку и общий для всех потоков счетчик оставшихся страниц декрементируется. При равенстве счетчика нулю поиск останавливается.

\aasection{Примеры работы программы}{sec:demo}

На рисунках ~\ref{img:in}-~\ref{img:out} представлен пример работы программы.

\FloatBarrier
\includeimage
{in} % Имя файла без расширения (файл должен быть расположен в директории inc/img/)
{f} % Обтекание (без обтекания)
{h} % Положение рисунка (см. figure из пакета float)
{\textwidth} % Ширина рисунка
{Ввод входных данных} % Подпись рисунка
\FloatBarrier
\FloatBarrier
\includeimage
{out} % Имя файла без расширения (файл должен быть расположен в директории inc/img/)
{f} % Обтекание (без обтекания)
{h} % Положение рисунка (см. figure из пакета float)
{\textwidth} % Ширина рисунка
{Полученная папка с рецептами} % Подпись рисунка
\FloatBarrier


\aasection{Тестирование}{sec:tests}

В таблице~\ref{tbl:tests} представлены функциональные тесты для разработанного ПО. Все тесты пройдены успешно. Порядок ввода входных данных: имя папки, количество рецептов, количество потоков.

\begin{longtable}{|p{.2\textwidth - 2\tabcolsep}|p{.33\textwidth - 2\tabcolsep}|p{.24\textwidth - 2\tabcolsep}|p{.23\textwidth - 2\tabcolsep}|}
    \caption{Функциональные тесты}\label{tbl:tests} \\\hline
    № теста & Входные данные & Полученные выходные данные & Ожидаемые выходные данные                                          \\\hline
    \endfirsthead
    \caption{Функциональные тесты (продолжение)} \\\hline
    № теста & Входные данные & Полученные выходные данные  & Ожидаемые выходные данные                                                 \\\hline
    \endhead
    \endfoot
    1                                           & recipes 10 0 & recipes/ с 10 рецептами & recipes/ с 10 рецептами \\\hline
    2                                           & recipes 10 4 & recipes/ с 10 рецептами & recipes/ с 10 рецептами \\\hline
    3                                           & recipes 10 64 & recipes/ с 10 рецептами & recipes/ с 10 рецептами \\\hline
    \end{longtable}

\aasection{Описание исследования}{sec:study}

В ходе исследования требуется исследовать зависимость производительности разработанного ПО (в терминах количества обработанных страниц в единицу времени) от количества дополнительных потоков. Изменять количество дополнительных потоков от 0 (вычисление в основном потоке), до $4\cdot k$, где $k$ --- количество логических ядер используемой ЭВМ, по степеням числа 2. Количество логических ядер процессора, на котором проводилось исследование (13th Gen Intel(R) Core(TM) i5-13500H   2.60 GHz) равно 16, поэтому замеры проводились на интервале от 0 до 64 потоков.
\newpage

В таблице~\ref{tbl:b_log} приведены замеры зависимости количества секунд, потраченных на один рецепт от количества потоков, на рисунке ~\ref{fig:time} приведен график, построенный по таблице.

\begin{longtable}{|p{.33\textwidth - 2\tabcolsep}|p{.33\textwidth - 2\tabcolsep}|p{.34\textwidth - 2\tabcolsep}|}
    \caption{Таблица замеров времени обработки одной страницы от количества потоков}\label{tbl:b_log}
    \\
    \hline
    Количество потоков & Количество полученных за секунду рецептов\\
    \hline

    \hline
    \endhead
    \hline
    \endfoot
    \endlastfoot
    \hline
    0  & 0.677015 \\ \hline
    1 &0.689014 \\ \hline
    2 &0.675797 \\ \hline
    4 &0.618141 \\ \hline
    8 &0.586283 \\ \hline
    16 &0.564012 \\ \hline
    32 &0.585115 \\ \hline
    64 &0.602166 \\ \hline
\end{longtable}

\begin{figure}[h]
 \centering
 \includesvg[scale=0.7]{inc/img/plot.svg}
 \caption{График зависимости времени на один рецепт от количества потоков}
 \label{fig:time}
\end{figure}
\newpage

По результатам проведенного исследования сделан вывод о том, что параллельная обработка данных может давать прирост к производительности, но только когда количество потоков не больше чем количество логических ядер процессора. График показывает что при большем количестве потоков степень время выполнения увеличивается, так как степень количество параллельных вычислений не изменяется, а количество издержек на создание потоков увеличивается.

\aaunnumberedsection{ЗАКЛЮЧЕНИЕ}{sec:outro}

Цель работы достигнута. Решены все поставленные задачи:
\begin{itemize}
    \item анализ предметной области;
    \item разработка алгоритма обработки данных;
    \item создание ПО, реализующего разработанный алгоритм;
    \item исследование характеристик созданного ПО.
\end{itemize}
