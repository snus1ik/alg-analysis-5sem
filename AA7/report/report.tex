\documentclass{bmstu}
\usepackage{multirow}
\usepackage{threeparttable}
\usepackage[clean]{svg}
\usepackage{adjustbox}

% пакеты
\usepackage{csquotes}
\usepackage[utf8]{inputenc}
\usepackage[T1]{fontenc}

% Библиография
\NewBibliographyString{accessmode}
\DeclareFieldFormat{url}{\bibstring{accessmode}\addcolon\space\url{#1}}
\DeclareFieldFormat{title}{#1}
\DefineBibliographyStrings{russian}{
accessmode={Режим доступа:},
urlseen={дата обращения:}
}
% скобки после пунктов
\setenumerate[1]{label={ \arabic*)}}
\usepackage{enumitem}

% математика в листингах
\lstset{
mathescape=true
}

% \FloatBarrier
\usepackage{placeins}
\usepackage{svg}

% $ в листингах
\newcommand{\dlr}{\mbox{\textdollar}}

% Тире в itemize
\renewcommand\labelitemi{---}

% \foreach
\usepackage{pgffor}

% Длинные таблицы
\usepackage{longtable}

% Картинки
\newcommand{\img}[3] {
\begin{figure}[t!]
    \center{\includegraphics[height=#1]{inc/img/#2}}
    \caption{#3}
    \label{img:#2}
\end{figure}
}
\newcommand{\imgw}[3] {
\begin{figure}[h!]
    \center{\includegraphics[width=#1]{inc/img/#2}}
    \caption{#3}
    \label{img:#2}
\end{figure}
}
\newcommand{\boximg}[3] {
\begin{figure}[h]
    \center{\fbox{\includegraphics[height=#1]{inc/img/#2}}}
    \caption{#3}
    \label{img:#2}
\end{figure}
}

% АА в сокращенном виде

\newcommand{\aasection}[2] {
    \vspace{20mm}
    {\let\clearpage\relax \chapter{#1}\label{#2}}
}

\newcommand{\aaunnumberedsection}[2] {
    \vspace{20mm}
    \addcontentsline{toc}{chapter}{#1}
    {\let\clearpage\relax \chapter*{#1}\label{#2}}
}




\bibliography{biblio}

\begin{document}
\renewcommand{\thelstlisting}{\arabic{lstlisting}}
\lstset{%
	language=python,
	basicstyle=\small\sffamily,
	numbers=left,
	stepnumber=1,
	numbersep=5pt,
	frame=single,
	tabsize=4,
	captionpos=t,
	breaklines=true,
	breakatwhitespace=true,
	escapeinside={\#*}{*)},
	backgroundcolor=\color{white},
}

\begin{titlepage}
	\newgeometry{pdftex, left=2cm, right=2cm, top=2.5cm, bottom=2.5cm}
	\fontsize{12pt}{12pt}\selectfont
	\noindent \begin{minipage}{0.15\textwidth}
		\includegraphics[width=\linewidth]{img/b_logo.jpg}
	\end{minipage}
	\noindent\begin{minipage}{0.9\textwidth}\centering
		\textbf{Министерство науки и высшего образования Российской Федерации}\\
		\textbf{Федеральное государственное бюджетное образовательное учреждение высшего образования}\\
		\textbf{«Московский государственный технический университет имени Н. Э.~Баумана}\\
		\textbf{(национальный исследовательский университет)»}\\
		\textbf{(МГТУ им. Н. Э.~Баумана)}
	\end{minipage}

	\noindent\rule{18cm}{3pt}
	\newline\newline
	\noindent ФАКУЛЬТЕТ $\underline{\text{«Информатика и системы управления»~~~~~~~~~~~~~~~~~~~~~~~~~~~~~~~~~~~~~~~~~~~~~~~~~~~~~~~}}$ \newline\newline
	\noindent КАФЕДРА $\underline{\text{«Программное обеспечение ЭВМ и информационные технологии»~~~~~~~~~~~~~~~~~~~~~~~}}$\newline\newline\newline\newline\newline\newline\newline


	\begin{center}
		\noindent\begin{minipage}{1.3\textwidth}\centering
		\Large\textbf{   ~~~ Лабораторная работа №7}\newline
		\textbf{по дисциплине <<Анализ Алгоритмов>>}\newline\newline\newline
		\end{minipage}
	\end{center}

	\noindent\textbf{Тема} 			$\underline{\text{Графовые представления}}$\newline\newline
	\noindent\textbf{Студент} 		$\underline{\text{Куликов Е. А.}}$\newline\newline
	\noindent\textbf{Группа} 		$\underline{\text{ИУ7-56Б}}$\newline\newline
	\noindent\textbf{Преподаватель} $\underline{\text{Волкова Л. Л.}}$\newline

	\begin{center}
		\vfill
		Москва~---~\the\year
		~г.
	\end{center}
	\restoregeometry
\end{titlepage}

\renewcommand{\contentsname}{СОДЕРЖАНИЕ}
\tableofcontents
\setcounter{page}{2}

% Введение
\chapter*{ВВЕДЕНИЕ}
\addcontentsline{toc}{chapter}{ВВЕДЕНИЕ}
\addcontentsline{toc}{chapter}{ВВЕДЕНИЕ}

\parЦелью данной лабораторной работы является исследование графовых моделей, для достижения поставленной цели необходимо решить следующие задачи:

\begin{itemize}
	\item[---] проанализировать предметную область;
	\item[---] построить информационный граф;
	\item[---] построить граф информационной истории;
        \item[---] построить граф управления;
        \item[---] построить граф операционной истории;
        \item[---] сделать вывод о применимости графовых моделей к данной задаче.
\end{itemize}

\chapter{Входные данные}
Входные данные --- фрагмент кода ЛР5, выходные --- графовые модели:
\begin{center}
\captionsetup{justification=raggedright,singlelinecheck=off}
\begin{lstlisting}[label=lst:listing, language=c, numbers=none, caption=Фрагмент кода]
while (true) {
        if (after_write_queue.empty())
            continue;
        std::shared_ptr<task> t;
        {
            std::unique_lock lock(mtx_after_write);
            t = after_write_queue.front();
            after_write_queue.pop();
            t->t.start_destr = std::chrono::high_resolution_clock::now();
        }
        std::ostringstream oss;
        oss << "[";
	for (int i = 0; i < t->recipe_ingredients.size(); i++) {
            oss << "{\"name\": \"";
            oss << t->recipe_ingredients[i];
            oss << "\"},";
        }
	std::string steps = oss.str();
	sql = "INSERT INTO RECIPE VALUES(" + std::to_string(n_tasks) + ", " + std::to_string(t->issue_id) +
            ", " + "'" + t->filename + "'" + ", " + "'" + t->recipe_title + "'" +
                ", " + "'" + steps + "'" +
                    ", " + "'" + t->recipe_img_url + "');";
        char *message_error;
        int rc = sqlite3_exec(db, sql.c_str(), nullptr, nullptr, &message_error);
        if (rc != SQLITE_OK) {
            std::cout << "SQL error: " << message_error << std::endl;
            sqlite3_free(message_error);
        }
	if (t->is_last)
            break;
}
\end{lstlisting}
\end{center}

\chapter{Описание исследования}
На рисунках \ref{fig:IG} --- \ref{fig:OI_GU} представлены полученные графовые модели:

\begin{figure}[h]
	\centering
        \vspace{0pt}
	\includegraphics[scale=0.35]{img/IG.jpg}
	\caption{Информационны граф}
 \vspace*{3in}
	\label{fig:IG}
\end{figure}

\begin{figure}[h]
	\centering
        \vspace{0pt}
	\includegraphics[scale=0.35]{img/II.jpg}
	\caption{Информационная история}
 \vspace*{3in}
	\label{fig:II}
\end{figure}

\begin{figure}[h]
	\centering
        \vspace{0pt}
	\includegraphics[scale=0.35]{img/OI_GU.jpg}
	\caption{Операционная история и граф управления}
 \vspace*{3in}
	\label{fig:OI_GU}
\end{figure}

Из полученных результатов сделан вывод о том, что графовые модели могут быть применимы для описания данного фрагмента кода.

\chapter*{ЗАКЛЮЧЕНИЕ}
\addcontentsline{toc}{chapter}{ЗАКЛЮЧЕНИЕ}
Цель работы достигнута, решены все поставленные задачи:
\begin{itemize}
	\item[---] проанализирована предметную область;
	\item[---] построен информационный граф;
	\item[---] построен граф информационной истории;
        \item[---] построен граф управления;
        \item[---] построен граф операционной истории;
        \item[---] сделан вывод о применимости графовых моделей к данной задаче.
\end{itemize}

\end{document}