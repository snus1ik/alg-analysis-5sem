% Содержимое отчета по курсу Анализ алгоритмов

\aaunnumberedsection{ВВЕДЕНИЕ}{sec:intro}

Цель работы --- Получить навык организации параллельных вычислений по конвейерному принципу. 

Задачи работы: 
\begin{itemize}
    \item анализ предметной области;
    \item разработка алгоритма конвейерной обработки данных;
    \item создание ПО, реализующего разработанный алгоритм;
    \item исследование характеристик созданного ПО.
\end{itemize}

\aasection{Входные и выходные данные}{sec:input-output}

Входными данными программы является название папки, в которой находятся html-файлы с рецептами, полученные в результате работы программы из предыдущей лабораторной работы. Выходными данными программы являются: база данных, содержащая информацию о рецептах --- порядковый номер, имя файла, заголовок рецепта, ингредиенты и способ приготовления в формате JSON, а также ссылку на основную картинку страницы.

\aasection{Преобразование входных данных в выходные}{sec:algorithm}
 
Программа считывает название папки с рецептами и обрабатывает каждый рецепт по конвейерному принципу: создаются 5 рабочих потоков --- поток, генерирующий заявки на обработку (создает структуру, в которой указан путь к файлу с рецептом); поток, выполняющий чтение данных из файла; поток, выполняющих поиск информации в файле; поток, выполняющий запись информации в базу данных $sqlite$ с использованием библиотеки $sqlite3$~\cite{sqlite_lib}; поток, выполняющий логирование и уничтожение заявки.
\par После обработки в каждом из потоков заявка попадает в одну из четырех очередей, в которой ожидает обработки следующим потоком.
\par Последний поток производит уничтожение очередной задачи и логирование --- подсчитывает среднее время обработки задачи в каждом потоке, а также время ожидания в каждой из очередей.
\par Кроме этого в специальной структуре на каждом этапе обработки фиксируется время начала обработки каждой задачи в каждом потоке, а также время постановки задачи в каждую из очередей. Эта информация сохраняется последним потоком и после обработки всех заявок записывается в файл логирования в хронологическом порядке.

\aasection{Примеры работы программы}{sec:demo}

На рисунках~\ref{img:in}-~\ref{img:out} представлен пример работы программы.

\FloatBarrier
\includeimage
{in} % Имя файла без расширения (файл должен быть расположен в директории inc/img/)
{f} % Обтекание (без обтекания)
{h} % Положение рисунка (см. figure из пакета float)
{\textwidth} % Ширина рисунка
{Ввод входных данных} % Подпись рисунка
\FloatBarrier
\FloatBarrier
\includeimage
{out} % Имя файла без расширения (файл должен быть расположен в директории inc/img/)
{f} % Обтекание (без обтекания)
{h} % Положение рисунка (см. figure из пакета float)
{\textwidth} % Ширина рисунка
{Полученная база данных} % Подпись рисунка
\FloatBarrier


\aasection{Тестирование}{sec:tests}

В таблице~\ref{tbl:tests} представлены функциональные тесты для разработанного ПО. Все тесты пройдены успешно. Входные данные: имя папки.

\begin{longtable}{|p{.2\textwidth - 2\tabcolsep}|p{.33\textwidth - 2\tabcolsep}|p{.24\textwidth - 2\tabcolsep}|p{.23\textwidth - 2\tabcolsep}|}
    \caption{Функциональные тесты}\label{tbl:tests} \\\hline
    № теста & Входные данные & Полученные выходные данные & Ожидаемые выходные данные                                          \\\hline
    \endfirsthead
    \caption{Функциональные тесты (продолжение)} \\\hline
    № теста & Входные данные & Полученные выходные данные  & Ожидаемые выходные данные                                                 \\\hline
    \endhead
    \endfoot
    1                                           & recipes & база данных на 10 записей & база данных на 10 записей \\\hline
    2                                           & notrecipes & Directory doesnt exist & Directory doesnt exist \\\hline
    \end{longtable}

\aasection{Описание исследования}{sec:study}

В ходе исследования определялось среднее время ожидания заявки в каждой из очередей, время обработки в каждом из потоков и связь этих времен.
На рисунке~\ref{img:times} представлено среднее время обработки задачи в каждой стадии, а также среднее время ожидания в каждой из очередей.

\FloatBarrier
\includeimage
{times} % Имя файла без расширения (файл должен быть расположен в директории inc/img/)
{f} % Обтекание (без обтекания)
{h} % Положение рисунка (см. figure из пакета float)
{\textwidth} % Ширина рисунка
{Характеристики среднего времени обработки и ожидания} % Подпись рисунка
\FloatBarrier

По результатам проведенного исследования сделан вывод о том, что самыми долгими стадиями обработки являются чтение html-строки данных из файла, а также логирование и уничтожение заявки. Поэтому аналогично самое долгое время ожидания наблюдается в очередях перед чтением и перед логированием и уничтожением. 

\aaunnumberedsection{ЗАКЛЮЧЕНИЕ}{sec:outro}

Цель работы достигнута. Решены все поставленные задачи: 
\begin{itemize}
    \item анализ предметной области;
    \item разработка алгоритма конвейерной обработки данных;
    \item создание ПО, реализующего разработанный алгоритм;
    \item исследование характеристик созданного ПО.
\end{itemize}
